%\usepackage{parskip}
%\setlength{\parindent}{0pt}

% ------------------------------------------------------------------------------
% Comandi personalizzati
% ------------------------------------------------------------------------------

\newcommand{\titolo}{Generazione sicura di chiavi OpenPGP}
\newcommand{\sottotitolo}{Guida dettagliata all'uso}
\newcommand{\autore}{Aldo Latino}
\newcommand{\luogo}{Italy}
\newcommand{\data}{1 aprile 2018 -- 14 luglio 2019}
% \renewcommand{\thefootnote}{\fnsymbol{footnote}}
\usepackage{marginnote}
\newcommand{\nota}[1]{\marginnote{\raggedright\footnotesize\itshape#1}}
%\newcommand{\nota}[1]{\leavevmode\marginpar{\strut\raggedright\footnotesize\itshape#1}\strut}
%\newcommand{\nota}[1]{\leavevmode\marginpar{\raggedright\footnotesize\itshape\strut#1}\strut\par}

% Tutto maiuscoletto non spaziato
\newcommand{\allsc}[1]{%
    {\addfontfeature{RawFeature=+c2sc;+smcp;}\MakeUppercase{#1}}%
}

% Tutto maiuscoletto spaziato
\newcommand{\allscsp}[1]{%
    {\addfontfeature{LetterSpace=7,RawFeature=+c2sc;+smcp;}\MakeUppercase{#1}}%
}

% Maiuscoletto. La macro \sc già esiste in LaTeX ma non è racchiusa dentro un
% gruppo {}, per cui la ridefinisco.
\renewcommand{\sc}[1]{%
    {\addfontfeature{RawFeature=+smcp;}#1}%
}

% Maiuscoletto spaziato.
\newcommand{\scsp}[1]{%
    {\addfontfeature{LetterSpace=7,RawFeature=+smcp;}#1}%
}

% Punto centrale
\newcommand{\puntoc}{$\cdot$}

% ------------------------------------------------------------------------------
% Pacchetti
% ------------------------------------------------------------------------------

% Carico fontspec. Per compilare sarà necessario XeLaTeX.
\usepackage{fontspec}
\setmainfont{EB Garamond}[%
    RawFeature={+kern;+pnum;+onum;+ss06}% +ss06 attiva la Q swash.
]%

\usepackage{polyglossia}
\setmainlanguage{italian}
\PolyglossiaSetup{italian}{indentfirst=false}

% Carico geometry per poter definire pagina e margini.
\usepackage[%
    a4paper,%
    top=35mm,%
    bottom=30mm,%
    inner=25mm,%
    outer=35mm,%
    marginparsep=5mm,
    marginparwidth=20mm,
    headsep=11pt,%
]{geometry}

% Carico fancyhdr che mi permette di ridefinire testatine e piè di pagina.
% Le impostazioni si trovano nel file indice.tex.
\usepackage{fancyhdr}
\fancyhf{} % sets both header and footer to nothing
\pagestyle{fancy}
\lhead{\footnotesize{\sc{\autore}\enspace\puntoc\enspace\emph{\titolo}}}% A sinistra della testatina.
\rhead{\footnotesize{\thepage}}% A destra della testatina.

% Rimuove il numero di pagina dal piede della prima pagina di un capitolo.
\fancypagestyle{plain}{
    \renewcommand{\headrulewidth}{0pt}
    \fancyhf{}
}

% Carico emptypage per eliminare le testatine dalle pagine bianche
% alla fine di ogni capitolo.
% https://tex.stackexchange.com/questions/1681/how-to-remove-headers-and-footers-for-pages-between-chapters
\usepackage{emptypage}

\usepackage[]{microtype}
\usepackage{csquotes}
\usepackage{tabularx}
% Questo pacchetto crea dei filetti migliori nelle tabelle.
\usepackage{booktabs}

\usepackage[]{titlesec}

\titleformat{\section}% command
[block]% shape
{\centering\normalfont\scshape\color{rosso}}% format
{\thesection.\hspace{.5em}}% label
{0pt}% sep
{\normalsize}% before code
{}% after code

% Tolgo lo spazio prima del titolo di ogni sezione.
% Macro di titlesec.
\titlespacing*{\section}%
{0pt}%   Left
{24pt}%  Prima del titolo
{12pt}%   Dopo il titolo

\titleformat{\subsection}%
[hang]%
{\centering\normalfont\itshape\color{rosso}}%
{\thesubsection.\hspace{.5em}}%
{0pt}%
{\normalsize}%
{}

% Tolgo lo spazio prima del titolo di ogni sottosezione.
% Macro di titlesec.
\titlespacing*{\subsection}%
{0pt}%   Left
{24pt}%  Prima del titolo
{12pt}%   Dopo il titolo

% Carico titletoc che permette di togliere la numerazione
% dei capitoli e delle sezioni dall'indice.
\usepackage{titletoc}

% Sezioni nell'indice.
\titlecontents{section}%
[0em]% Spazio a sinistra.
{\vskip 4ex}% Codice prima della riga.
{\addfontfeature{RawFeature=+smcp;}\thecontentslabel.\hspace{1em}}% Formattazione delle parti numerate.
{}% Formattazione delle parti non numerate.
{\titlerule*[1pc]{}\addfontfeature{RawFeature=-pnum;}\contentspage}% I puntini dell'indice e il numero di pagina.
[]% Codice dopo la riga.

% Sottosezioni nell'indice.
\titlecontents{subsection}%
[1.7em]%
{\vskip 0ex}%
{\itshape\thecontentslabel.\hspace{1em}}% numbered sections formatting
{}% unnumbered sections formatting
{\titlerule*[1pc]{.}\addfontfeature{RawFeature=-pnum;}\contentspage}% Le parentesi graffe dopo \titlerule possono contenere . per i puntini nell'indice.


\usepackage[
  %backend=biber,
  bibstyle=authortitle,%alphabetic,%authoryear,
  citestyle=verbose,
  sorting=nty
]{biblatex}
\bibliography{bibliografia}

% \usepackage{acronym}

\usepackage{graphicx}
\graphicspath{{img/}}

\usepackage{caption}
\captionsetup{
    font=footnotesize,%       Corpo delle didascalie
    labelsep=period,%         Separatore tra etichetta e testo della didascalia
    textformat=period,%       Mette un punto dopo la didascalia
    format=hang,%             Rientra il testo della didascalia rispetto alla parola Figura 1
    justification=justified,% Giustifica il testo della didascalia
    labelfont=sc,%            Mette in maiuscoletto l'etichetta della didascalia
}

\usepackage{wrapfig}

% Rimuove lo spazio tra le voci di una lista.
\usepackage{enumitem}
%\setlist[itemize]{noitemsep,nolistsep}
%\setlist[enumerate]{noitemsep,nolistsep}
\setlist{noitemsep}

\usepackage[italian]{varioref}

% \usepackage{xspace} % to get the spacing after macros right

\usepackage[%
	usenames,%
	dvipsnames%
]{xcolor}
\definecolor{rosso}{RGB}{237, 61, 74}
\definecolor{blu}{RGB}{8, 59, 125}
\definecolor{codice}{RGB}{99, 99, 99}

\usepackage{relsize}
\def\UrlFont{\relsize{-1}\ttfamily}
% Ri-definizione del comando \texttt con font ridotto
\let\oldtexttt\texttt
\renewcommand{\texttt}[1]{{\relsize{-.5}\oldtexttt{#1}}}

\usepackage{tabularx} % better tables
\setlength{\extrarowheight}{3pt} % increase table row height

\usepackage[italian]{varioref}

\usepackage{textcomp}
\usepackage{listings}
%\lstset{emph={trueIndex,root},emphstyle=\color{BlueViolet}}%\underbar} % for special keywords
\lstset{
  language=[LaTeX]Tex,
  morekeywords={PassOptionsToPackage,selectlanguage},
  basicstyle=\small\ttfamily\color{codice},
  keywordstyle=\color{RoyalBlue},%\bfseries,
  %identifierstyle=\color{NavyBlue},
  commentstyle=\color{Green}\ttfamily,
  stringstyle=\rmfamily,
  numbers=none,%left,%none
  numberstyle=\scriptsize,%\tiny
  stepnumber=5,
  numbersep=8pt,
  showstringspaces=false,
  breaklines=true,
  %frameround=ftff,
  %frame=single,
  belowcaptionskip=.75\baselineskip,
  upquote=true
  %frame=L
}

% \usepackage{showframe}

\usepackage{hyperref}
\hypersetup{%
  %draft, % hyperref's draft mode, for printing see below
  colorlinks=true,
  linktocpage=true,
  pdfstartpage=1,
  pdfstartview=FitV,
  % uncomment the following line if you want to have black links (e.g., for printing)
  %colorlinks=false, linktocpage=false, pdfstartpage=3, pdfstartview=FitV, pdfborder={0 0 0},%
  breaklinks=true,
  pageanchor=true,
  pdfpagemode=UseNone,
  % pdfpagemode=UseOutlines,
  plainpages=false,
  bookmarksnumbered,
  bookmarksopen=true,
  bookmarksopenlevel=1,
  hypertexnames=true,
  pdfhighlight=/O,
  % nesting=true,
  % frenchlinks,
  urlcolor=blu, linkcolor=blu, citecolor=Black, %pagecolor=RoyalBlue,%
  %urlcolor=Black, linkcolor=Black, citecolor=Black, %pagecolor=Black,%
  pdftitle={\titolo},
  pdfauthor={\autore},
  pdfsubject={\sottotitolo},
  pdfkeywords={Sicurezza, Crittografia, Privacy, GnuPG, OpenPGP},
  pdfcreator={XeLaTeX},
  pdfproducer={LaTeX with hyperref}%
}
