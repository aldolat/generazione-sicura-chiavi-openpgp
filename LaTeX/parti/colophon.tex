\thispagestyle{empty}
\begingroup
	\section*{Sommario}

	Generare una coppia di chiavi con GnuPG è un’operazione semplice, ma se
	abbiamo l’esigenza di crearle e proteggerle in un modo abbastanza sicuro, al
	riparo da sguardi indiscreti per le più svariate ragioni, allora bisogna
	prendere qualche precauzione. Sviluppatori di software, giornalisti,
	whistleblower e utenti che vivono in stati ficcanaso potrebbero aver bisogno
	di un ambiente sanificato. In questa guida vedremo una strada possibile per
	arrivare a generarle in modo abbastanza sicuro, soprattutto tentando di
	proteggere in modo diretto la chiave “master”, vale a dire quella che è il
	fondamento di tutto il mazzo di chiavi. Non si tratta di una guida per
	tutti, soprattutto perché nel tempo la gestione del portachiavi è onerosa in
	termini di cura e attenzione.\bigskip

	\vfill

	\footnotesize

	\noindent\href{https://creativecommons.org/licenses/by-sa/4.0/deed.it}{\includegraphics[width=0.09\textwidth]{cc-80x15}\\
	Creative Commons Attribuzione -- Condividi allo stesso modo 4.0
	Internazionale (CC BY-SA 4.0)}\bigskip

	\noindent La guida è stata pubblicata la prima volta su
	\href{https://github.com/aldolat/generazione-sicura-chiavi-openpgp}{GitHub}.\bigskip

	\noindent Creato con \LaTeX.

\endgroup