\section{Gli agenti}

Fermiamo \texttt{ssh-agent}:

\begin{lstlisting}
killall ssh-agent
\end{lstlisting}

Disabilitiamo \texttt{ssh-agent} permanentemente:

\begin{lstlisting}
sudo nano /etc/X11/Xsession.options
\end{lstlisting}

e commentiamo la riga aggiungendo un cancelletto \texttt{\#} all'inizio:

\begin{lstlisting}
# use-ssh-agent
\end{lstlisting}

Abilitiamo il supporto SSH in GnuPG aggiungendo la seguente riga in
\texttt{/home/user/.gnupg/gpg-agent.conf}:

\begin{lstlisting}
enable-ssh-support
\end{lstlisting}

Già che ci siamo aggiungiamo pure la scelta di quale \texttt{pinentry} usare:

\begin{lstlisting}
pinentry-program /usr/bin/pinentry-gtk-2
\end{lstlisting}

Alternative sono, ad esempio:

\begin{lstlisting}
pinentry-program /usr/bin/pinentry-curses
pinentry-program /usr/bin/pinentry-tty
pinentry-program /usr/bin/pinentry-qt
pinentry-program /usr/bin/pinentry-gnome3
\end{lstlisting}

Riavviamo l'agente di GnuPG:

\begin{lstlisting}
gpg-connect-agent killagent /bye
gpg-connect-agent /bye
\end{lstlisting}

Inseriamo la riga seguente in \texttt{/home/user/.bashrc}:

\begin{lstlisting}
# You should always add the following lines to your .bashrc or whatever initialization file is used for all shell invocations.
# @link https://gnupg.org/documentation/manuals/gnupg/Invoking-GPG_002dAGENT.html#Invoking-GPG_002dAGENT
GPG_TTY=$(tty)
export GPG_TTY

# Launch gpg-agent
gpg-connect-agent /bye

# When using SSH support, use the current TTY for passphrase prompts
gpg-connect-agent updatestartuptty /bye > /dev/null

# Point the SSH_AUTH_SOCK to the one handled by gpg-agent
if [ -S $(gpgconf --list-dirs agent-ssh-socket) ]; then
  export SSH_AUTH_SOCK=$(gpgconf --list-dirs agent-ssh-socket)
else
  echo "$(gpgconf --list-dirs agent-ssh-socket) doesn't exist. Is gpg-agent running ?"
fi
\end{lstlisting}

Facciamo rileggere questo file:

\begin{lstlisting}
source .bashrc
\end{lstlisting}

Infine, se non l'abbiamo già fatto, assicuriamoci di usare \texttt{gpg2} invece
della versione 1:

\begin{lstlisting}
echo "alias gpg=gpg2" >> /home/user/.bash_aliases
\end{lstlisting}
