\section{Eliminare la chiave principale}

Se abbiamo scelto di eliminare la chiave principale dal disco, dobbiamo
anzitutto verificare che versione di GnuPG stiamo usando, perché a seconda della
versione bisognerà procedere in due modi diversi:

\begin{itemize}
   \item \emph{se abbiamo GnuPG versione 2.1 o successiva:} è sufficiente
   cancellare il file corrispondente alla chiave master dalla directory
   \texttt{/home/user/.gnupg/private-keys-v1.d};
   \item \emph{se abbiamo GnuPG precedente alla versione 2.1:} dopo aver fatto
   il backup delle sottochiavi, bisogna dapprima cancellare tutta la chiave
   (master e sottochiavi) e poi reimportare solo le sottochiavi.
\end{itemize}

Nonostante all'inizio della guida ho specificato di usare solo la versione di
GnuPG 2.1 o successiva, vediamo comunque i due procedimenti nel dettaglio.

\subsection{GnuPG versione 2.1 o successiva}

GnuPG, dalla versione 2.1 in su, deposita le chiavi private nella directory
\texttt{/home/user/.gnupg/private-keys-v1.d/} e non più nel file
\texttt{/home/user/.gnupg/secring.gpg}. Perciò, se avete una chiave master e 3
sottochiavi, dovreste avere 4 file con estensione \texttt{.key} in quella
directory:

\begin{lstlisting}
C6D1CD1D12CBBC8FA14D30004EAF381803A72597.key
135159EDEFB9FCB6062F9DE0E21FD90CB07DA041.key
C74D47329D5D224B0716879DCF3A4EC2D8951C53.key
30DEE8A060F3562CD6F1384E0F883D65477E596A.key
\end{lstlisting}

Per identificare quale di questi file è la chiave master, dare nel terminale
questo comando:

\begin{lstlisting}
gpg --with-keygrip --list-key 0x9F676B5A4B6E6777
\end{lstlisting}

Siccome è stato fatto il backup della directory di GnuPG come descritto al
paragrafo \vref{backup-directory-gnupg}, potete eliminare il file \texttt{.key}
corrispondente alla chiave master. Per ogni evenienza, prima di cancellare il
file, verificate che nel backup ci sia il file che state per cancellare. Il
comando per eliminare il file è:

\begin{lstlisting}
cp rm /home/user/.gnupg/private-keys-v1.d/C6D1CD1D12CBBC8FA14D30004EAF381803A72597.key
\end{lstlisting}

oppure usate un programma come \texttt{shred} per eliminarlo in modo
sicuro:

\begin{lstlisting}
shred -vuzn25 /home/user/.gnupg/private-keys-v1.d/C6D1CD1D12CBBC8FA14D30004EAF381803A72597.key
\end{lstlisting}

Per le opzioni di \texttt{shred} vedi la tabella \vref{table:shred}.

\begin{table}
    \centering
    \begin{tabularx}{.6\textwidth}{c l}
        \toprule
        \textsc{Opzione} & \textsc{Descrizione} \\
        \midrule
        \texttt{v} & Show progress. \\
        \texttt{u} & Truncate and remove file after overwriting. \\
        \texttt{z} & Add a final overwrite with zeros to hide shredding. \\
        \texttt{n} & Overwrite $n$ times instead of the default (3). \\
        \bottomrule
    \end{tabularx}
    \caption{Le opzioni di shred usate in questo \textit{how-to}}
    \label{table:shred}
\end{table}

Se avete fatto da poco la migrazione a GnuPG versione 2.1 o superiore, il
vecchio file \texttt{secring.gpg} che conteneva le chiavi private potrebbe
ancora averla al suo interno. Per cui, se la dimensione del file è zero, il file
non contiene più nulla; altrimenti, se la dimensione del file è maggiore di
zero, cancellatelo del tutto.

Quando in futuro vi servirà usare la chiave master (ad esempio, per firmare una
chiave altrui), potrete semplicemente prendere dal backup il file \texttt{.key}
della chiave master e copiarlo nella directory \texttt{private-keys-v1.d} di
lavoro. Una volta terminata l'esigenza, potrete eliminare nuovamente il file.

\subsection{GnuPG precedente alla versione 2.1}

Se avete questa versione di GnuPG, bisognerà procedere con questi passi:

\begin{enumerate}
   \item Essere sicuri di avere il backup funzionante delle chiavi, come
   spiegato al paragrafo \vref{backup-chiavi};
   \item Rimuovere la chiave privata completa (cioè master e sottochiavi);
   \item Reimportare dal backup solo le sottochiavi.
\end{enumerate}

La chiave completa si elimina con:

\begin{lstlisting}
gpg --delete-secret-keys 0x9F676B5A4B6E6777
\end{lstlisting}

Le sottochiavi si importano con:

\begin{lstlisting}
gpg --import 0xE7E0CAF69114F367
gpg --import 0x3C91B3682F3AC08A
gpg --import 0x465ED456AFBE0F10
\end{lstlisting}

Vediamo quindi la situazione:

\begin{lstlisting}
gpg --list-secret-keys
\end{lstlisting}

Ora, \emph{se avete spostato le sottochiavi nella YubiKey}, la risposta sarà:

\begin{lstlisting}
sec#  rsa4096/0x9F676B5A4B6E6777 2018-04-01 [C] [expires: 2021-03-31]
      Key fingerprint = 4915 3275 9708 0147 25CF  E3A7 9F67 6B5A 4B6E 6777
      Keygrip = C6D1CD1D12CBBC8FA14D30004EAF381803A72597
uid                   [ultimate] Mario Rossi <mario.rossi@example.com>
ssb>  rsa2048/0xE7E0CAF69114F367 2018-04-01 [S] [expires: 2019-04-01]
      Keygrip = 135159EDEFB9FCB6062F9DE0E21FD90CB07DA041
ssb>  rsa2048/0x3C91B3682F3AC08A 2018-04-01 [E] [expires: 2019-04-01]
      Keygrip = C74D47329D5D224B0716879DCF3A4EC2D8951C53
ssb>  rsa2048/0x465ED456AFBE0F10 2018-04-01 [A] [expires: 2019-04-01]
      Keygrip = 30DEE8A060F3562CD6F1384E0F883D65477E596A
\end{lstlisting}

mentre, \emph{se avete deciso di tenere le sottochiavi nel disco}, la risposta
sarà:

\begin{lstlisting}
sec#  rsa4096/0x9F676B5A4B6E6777 2018-04-01 [C] [expires: 2021-03-31]
      Key fingerprint = 4915 3275 9708 0147 25CF  E3A7 9F67 6B5A 4B6E 6777
      Keygrip = C6D1CD1D12CBBC8FA14D30004EAF381803A72597
uid                   [ultimate] Mario Rossi <mario.rossi@example.com>
ssb   rsa2048/0xE7E0CAF69114F367 2018-04-01 [S] [expires: 2019-04-01]
      Keygrip = 135159EDEFB9FCB6062F9DE0E21FD90CB07DA041
ssb   rsa2048/0x3C91B3682F3AC08A 2018-04-01 [E] [expires: 2019-04-01]
      Keygrip = C74D47329D5D224B0716879DCF3A4EC2D8951C53
ssb   rsa2048/0x465ED456AFBE0F10 2018-04-01 [A] [expires: 2019-04-01]
      Keygrip = 30DEE8A060F3562CD6F1384E0F883D65477E596A
\end{lstlisting}
