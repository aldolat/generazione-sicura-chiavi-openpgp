\section{Trasferire le chiavi OpenPGP su YubiKey}

Se avete un token \textsc{usb} come la YubiKey, potete trasferire le sottochiavi
in essa. L'operazione è molto semplice.

Prima di andare avanti, quando dico ``trasferire'' intendo proprio spostare una
cosa da un posto a un altro. Nel nostro caso, intendo spostare le sottochiavi
dal disco fisso alla YubiKey.

Ricordo che trasferire le sottochiavi nel token \textsc{usb} è una
\emph{operazione a senso unico e non è possibile tornare indietro}, cioè non
si possono trasferire le sottochiavi dal token al nostro disco fisso. Solo se
abbiamo un backup delle chiavi possiamo ripristinare la situazione allo stato
prima del trasferimento. Per cui \emph{non procedete se non avete fatto il
backup}.

Diamo uno sguardo alla situazione attuale:

\begin{lstlisting}
gpg --list-secret-keys
\end{lstlisting}

\begin{lstlisting}
sec   rsa4096/0x9F676B5A4B6E6777 2018-04-01 [C] [expires: 2021-03-31]
      Key fingerprint = 4915 3275 9708 0147 25CF  E3A7 9F67 6B5A 4B6E 6777
      Keygrip = C6D1CD1D12CBBC8FA14D30004EAF381803A72597
uid                   [ultimate] Mario Rossi <mario.rossi@example.com>
ssb   rsa2048/0xE7E0CAF69114F367 2018-04-01 [S] [expires: 2019-04-01]
      Keygrip = 135159EDEFB9FCB6062F9DE0E21FD90CB07DA041
ssb   rsa2048/0x3C91B3682F3AC08A 2018-04-01 [E] [expires: 2019-04-01]
      Keygrip = C74D47329D5D224B0716879DCF3A4EC2D8951C53
ssb   rsa2048/0x465ED456AFBE0F10 2018-04-01 [A] [expires: 2019-04-01]
      Keygrip = 30DEE8A060F3562CD6F1384E0F883D65477E596A
\end{lstlisting}

Inseriamo il token nella porta \textsc{usb} e controlliamo che venga vista da
GnuPG:

\begin{lstlisting}
gpg --card-status
\end{lstlisting}

Se restituisce:

\begin{lstlisting}
gpg: error getting version from 'scdaemon': No SmartCard daemon
gpg: OpenPGP card not available: No SmartCard daemon
\end{lstlisting}

installiamo \texttt{scdaemon}:

\begin{lstlisting}
sudo apt-get install pcscd pcsc-tools scdaemon
\end{lstlisting}

e quindi di nuovo:

\begin{lstlisting}
gpg --card-status
\end{lstlisting}

che restituirà qualcosa del tipo:

\begin{lstlisting}
Reader ...........: Yubico Yubikey 4 OTP U2F CCID 00 00
Application ID ...: XXXXXXXXXXXXXXXXXXXXXXXXXXXXXXXX
Version ..........: 2.1
Manufacturer .....: Yubico
Serial number ....: XXXXXXXX
Name of cardholder: [not set]
Language prefs ...: [not set]
Sex ..............: unspecified
URL of public key : [not set]
Login data .......: [not set]
Signature PIN ....: not forced
Key attributes ...: rsa2048 rsa2048 rsa2048
Max. PIN lengths .: 127 127 127
PIN retry counter : 3 0 3
Signature counter : 0
Signature key ....: [none]
Encryption key....: [none]
Authentication key: [none]
General key info..: [none]
\end{lstlisting}

Notate come ci siano i tre campi dedicati alla conservazione delle tre
sottochiavi che abbiamo creato:

\begin{lstlisting}
Signature key ....: [none]
Encryption key....: [none]
Authentication key: [none]
\end{lstlisting}

Qui sposteremo le nostre tre sottochiavi. Iniziamo la procedura.

Entriamo in modalità modifica:

\begin{lstlisting}
gpg --card-edit
\end{lstlisting}

Cambiamo anzitutto i \textsc{pin}. I \textsc{pin} predefiniti di fabbrica sono:

\begin{itemize}
    \item Admin PIN (8 cifre): \texttt{12345678}
    \item PIN (6 cifre): \texttt{123456}
\end{itemize}

Entriamo in modalità amministratore dando:

\begin{lstlisting}
admin
\end{lstlisting}

\begin{lstlisting}
passwd
\end{lstlisting}

Dal menu scegliere l'opzione da seguire, vale a dire \texttt{Change PIN} e, una
volta finito, \texttt{Change Admin PIN}. Quindi alla fine salvare con
\texttt{save} e uscire con \texttt{quit}.

Una volta aggiornati i due \textsc{pin}, passare al trasferimento delle chiavi.
Dal prompt del terminale dare:

\begin{lstlisting}
gpg --edit-key 0x9F676B5A4B6E6777
\end{lstlisting}

che restituirà:

\begin{lstlisting}
gpg (GnuPG) 2.1.11; Copyright (C) 2016 Free Software Foundation, Inc.
This is free software: you are free to change and redistribute it.
There is NO WARRANTY, to the extent permitted by law.

Secret key is available.

sec  rsa4096/0x9F676B5A4B6E6777
     created: 2018-04-01  expires: 2021-03-31  usage: C
     trust: ultimate      validity: ultimate
ssb  rsa2048/0xE7E0CAF69114F367
     created: 2018-04-01  expires: 2019-04-01  usage: S
ssb  rsa2048/0x3C91B3682F3AC08A
     created: 2018-04-01  expires: 2019-04-01  usage: E
ssb  rsa2048/0x465ED456AFBE0F10
     created: 2018-04-01  expires: 2019-04-01  usage: A
[ultimate] (1). Mario Rossi <mario.rossi@example.com>
\end{lstlisting}

Selezioniamo la prima sottochiave con:

\begin{lstlisting}
key 1
\end{lstlisting}

che mostrerà la chiave selezionata con un asterisco \texttt{*}:

\begin{lstlisting}
sec  rsa4096/0x9F676B5A4B6E6777
     created: 2018-04-01  expires: 2021-03-31  usage: C
     trust: ultimate      validity: ultimate
ssb* rsa2048/0xE7E0CAF69114F367
     created: 2018-04-01  expires: 2019-04-01  usage: S
ssb  rsa2048/0x3C91B3682F3AC08A
     created: 2018-04-01  expires: 2019-04-01  usage: E
ssb  rsa2048/0x465ED456AFBE0F10
     created: 2018-04-01  expires: 2019-04-01  usage: A
[ultimate] (1). Mario Rossi <mario.rossi@example.com>
\end{lstlisting}

Trasferiamo la chiave con:

\begin{lstlisting}
keytocard
\end{lstlisting}

Deselezioniamo quindi la chiave 1:

\begin{lstlisting}
key 1
\end{lstlisting}

L'asterisco verrà tolto. Quindi procedere con lo stesso metodo per le restanti
due chiavi. Qui indico di seguito i comandi in sequenza:

\begin{lstlisting}
key 2
keytocard
key 2
key 3
keytocard
key 3
\end{lstlisting}

Alla fine salviamo e usciamo:

\begin{lstlisting}
save
\end{lstlisting}

Una volta usciti al prompt dei comandi, diamo uno sguardo alla situazione:

\begin{lstlisting}
gpg --edit-key 0x9F676B5A4B6E6777
\end{lstlisting}

che restituirà:

\begin{lstlisting}
gpg (GnuPG) 2.1.11; Copyright (C) 2016 Free Software Foundation, Inc.
This is free software: you are free to change and redistribute it.
There is NO WARRANTY, to the extent permitted by law.

Secret key is available.

sec  rsa4096/0x9F676B5A4B6E6777
     created: 2018-04-01  expires: 2021-03-31  usage: C
     trust: ultimate      validity: ultimate
ssb> rsa2048/0xE7E0CAF69114F367
     created: 2018-04-01  expires: 2019-04-01  usage: S
ssb> rsa2048/0x3C91B3682F3AC08A
     created: 2018-04-01  expires: 2019-04-01  usage: E
ssb> rsa2048/0x465ED456AFBE0F10
     created: 2018-04-01  expires: 2019-04-01  usage: A
[ultimate] (1). Mario Rossi <mario.rossi@example.com>
\end{lstlisting}

Come si vede abbiamo il segno \texttt{>} dopo ogni \texttt{ssb}, segno che le
sottochiavi sono nel token. La chiave principale invece è ancora nel nostro
disco. Date anche uno sguardo a com'è cambiata la YubiKey dando:

\begin{lstlisting}
gpg --card-status
\end{lstlisting}

Potremmo finire qua se abbiamo scelto di non cancellare la chiave principale dal
disco.
