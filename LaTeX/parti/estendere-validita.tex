\chapter{Estendere la validità delle nostre chiavi}

Se abbiamo scelto una validità a tempo per le nostre chiavi, periodicamente
potremo estenderne la validità se ancora sono valide per noi. Teniamo presente
che possiamo rinnovare la validità di una chiave già scaduta. Se abbiamo scelto
di togliere la chiave principale dal nostro disco, dovremo importarla
nuovamente, come descritto al paragrafo \vref{importare-chiave}. Se abbiamo
scelto, poi, di tenere le sottochiavi nel token \textsc{usb}, dovremo inserirlo
in una porta \textsc{usb}. Iniziamo quindi la procedura di rinnovo:

\begin{lstlisting}
gpg --edit-key 0x9F676B5A4B6E6777
\end{lstlisting}

\begin{lstlisting}
sec  rsa4096/0x9F676B5A4B6E6777
     created: 2018-04-01  expires: 2021-03-31  usage: C
     trust: ultimate      validity: ultimate
ssb  rsa2048/0xE7E0CAF69114F367
     created: 2018-04-01  expires: 2019-04-01  usage: S
ssb  rsa2048/0x3C91B3682F3AC08A
     created: 2018-04-01  expires: 2019-04-01  usage: E
ssb  rsa2048/0x465ED456AFBE0F10
     created: 2018-04-01  expires: 2019-04-01  usage: A
[ultimate] (1). Mario Rossi <mario.rossi@example.com>
\end{lstlisting}

I passi di rinnovo per ogni chiave sono i seguenti:

\begin{enumerate}
    \item selezione della sottochiave da rinnovare;
    \item invio del comando \texttt{expire};
    \item scelta della durata della validità;
    \item deselezione della sottochiave.
\end{enumerate}

Nel nostro caso, essendo tre le chiavi, questi i passaggi tutti di seguito:

\begin{lstlisting}
key 1 (per la prima sottochiave; key 2 per la seconda; key 3 per la terza)
   expire (e seguire le istruzioni)
      1y (per un anno di validita)
key 1 (deseleziona)
key 2 (seleziona la seconda sottochiave)
   expire
      1y
key 2 (deseleziona)
key 3 (seleziona la terza sottochiave)
   expire
      1y
check (verifichiamo che tutto sia stato eseguito)
save
\end{lstlisting}
