\section{Generazione delle chiavi}

Con questa guida genereremo la nostra chiave privata che avrà questo schema:

\begin{itemize}
    \item \emph{Chiave master (o principale)} [C -- Certification/Certificazione]
    \begin{itemize}
        \item \emph{Sottochiave per firma} [S -- Signing/Firma]
        \item \emph{Sottochiave per cifratura} [E -- Encryption/Cifratura]
        \item \emph{Sottochiave per autenticazione} [A --
    Authentication/Autenticazione]
    \end{itemize}
\end{itemize}

Per la lunghezza delle chiavi avremo la chiave primaria a 4096 bit
mentre le sottochiavi a 2048 bit.\footnote{Per ulteriori informazioni su questo
dibattito della lunghezza delle chiavi si può vedere: \cite{gnupg:keylength} e
anche \cite{yubico:keylength}}

\subsection{Prerequisiti}

\begin{enumerate}
 \item È necessario lavorare su un sistema avviato con una distribuzione Linux
 in modalità \textit{live}. È preferibile una
 \href{https://tails.boum.org/index.it.html}{Tails}, ma va bene una qualunque.
 La distribuzione va avviata isolandola da qualsiasi rete.
 \item È necessario usare GnuPG versione 2.1 o successiva.
 \item È necessario un token \textsc{usb} come, ad esempio, una YubiKey, nel
 caso si decida di spostare le sottochiavi su questo supporto.
\end{enumerate}

Qualora il comando \texttt{gpg} lanci GnuPG versione 1, accertarsi che
la versione 2 sia installata e aggiungere questa riga al seguente file:
\texttt{/home/user/.bash\_aliases}:

\begin{lstlisting}
alias gpg='gpg2'
\end{lstlisting}

Questa scorciatoia permetterà l'avvio di GnuPG versione 2 ogni volta
che digitiamo \texttt{gpg}.

\subsection{Generare le chiavi}

Sul sistema di generazione delle chiavi accertarsi di avere GnuPG versione 2.1 o
successiva:

\begin{lstlisting}
gpg --version
\end{lstlisting}

Iniziare quindi la procedura con:

\begin{lstlisting}
gpg --expert --full-gen-key
\end{lstlisting}

che darà:

\begin{lstlisting}
Please select what kind of key you want:
   (1) RSA and RSA (default)
   (2) DSA and Elgamal
   (3) DSA (sign only)
   (4) RSA (sign only)
   (7) DSA (set your own capabilities)
   (8) RSA (set your own capabilities)
   (9) ECC and ECC
  (10) ECC (sign only)
  (11) ECC (set your own capabilities)
Your selection?
\end{lstlisting}

Scegliamo \texttt{8}:

\begin{lstlisting}
Possible actions for a RSA key: Sign Certify Encrypt Authenticate
Current allowed actions: Sign Certify Encrypt

   (S) Toggle the sign capability
   (E) Toggle the encrypt capability
   (A) Toggle the authenticate capability
   (Q) Finished

Your selection?
\end{lstlisting}

Come si vede le attuali capacità della chiave che stiamo creando sono:
\texttt{Sign Certify Encrypt}. Rimuoviamo le capacità di firma premendo
\texttt{S}:

\begin{lstlisting}
Possible actions for a RSA key: Sign Certify Encrypt Authenticate
Current allowed actions: Certify Encrypt

   (S) Toggle the sign capability
   (E) Toggle the encrypt capability
   (A) Toggle the authenticate capability
   (Q) Finished

Your selection?
\end{lstlisting}

Rimuoviamo le capacità di cifratura premendo \texttt{E}:

\begin{lstlisting}
Possible actions for a RSA key: Sign Certify Encrypt Authenticate
Current allowed actions: Certify

   (S) Toggle the sign capability
   (E) Toggle the encrypt capability
   (A) Toggle the authenticate capability
   (Q) Finished

Your selection?
\end{lstlisting}

Ora la chiave primaria avrà solo la capacità di certificazione
(\texttt{Certify}) su altre chiavi.

Premiamo \texttt{Q}:

\begin{lstlisting}
RSA keys may be between 1024 and 4096 bits long.
What keysize do you want? (2048)
\end{lstlisting}


Inserire \texttt{4096}:

\begin{lstlisting}
Requested keysize is 4096 bits
Please specify how long the key should be valid.
         0 = key does not expire
      <n>  = key expires in n days
      <n>w = key expires in n weeks
      <n>m = key expires in n months
      <n>y = key expires in n years
Key is valid for? (0)
\end{lstlisting}


Inseriamo \texttt{3y} (3 anni):

\begin{lstlisting}
Key expires at mer 31 mar 2021 19:46:05 CEST
Is this correct? (y/N)
\end{lstlisting}

Premiamo \texttt{Y}:

\begin{lstlisting}
GnuPG needs to construct a user ID to identify your key.

Real name:
Email address:
Comment:
\end{lstlisting}


Inseriamo il nostro nome e cognome e quindi l'indirizzo email. Come commento
lasciamo il campo vuoto.

\begin{lstlisting}
You selected this USER-ID:
    "Mario Rossi <mario.rossi@example.com>"

Change (N)ame, (C)omment, (E)mail or (O)kay/(Q)uit?
\end{lstlisting}

Accettare con \texttt{O}:

\begin{lstlisting}
We need to generate a lot of random bytes. It is a good idea to perform
some other action (type on the keyboard, move the mouse, utilize the
disks) during the prime generation; this gives the random number
generator a better chance to gain enough entropy.
\end{lstlisting}

A seguire GnuPG chiederà di impostare la passphrase.

Impostiamo la passphrase.

\begin{lstlisting}
gpg: key 4B6E6777 marked as ultimately trusted
gpg: directory '/home/mariorossi/.gnupg/openpgp-revocs.d' created
gpg: revocation certificate stored as '/home/mariorossi/.gnupg/openpgp-revocs.d/491532759708014725CFE3A79F676B5A4B6E6777.rev'
public and secret key created and signed.

gpg: checking the trustdb
gpg: marginals needed: 3  completes needed: 1  trust model: PGP
gpg: depth: 0  valid:   1  signed:   0  trust: 0-, 0q, 0n, 0m, 0f, 1u
gpg: next trustdb check due at 2021-03-31
pub   rsa4096/4B6E6777 2018-04-01 [C] [expires: 2021-03-31]
      Key fingerprint = 4915 3275 9708 0147 25CF  E3A7 9F67 6B5A 4B6E 6777
uid         [ultimate] Mario Rossi <mario.rossi@example.com>
\end{lstlisting}

GnuPG mostra la chiave appena generata ed esce.

Come si vede, esiste solo la chiave principale con la sola capacità di
certificazione \texttt{[C]}. Bisogna ora generare le sottochiavi con:

\begin{lstlisting}
gpg --expert --edit-key 0x4B6E6777
\end{lstlisting}


\begin{lstlisting}
gpg (GnuPG) 2.1.11; Copyright (C) 2016 Free Software Foundation, Inc.
This is free software: you are free to change and redistribute it.
There is NO WARRANTY, to the extent permitted by law.

Secret key is available.

sec  rsa4096/4B6E6777
     created: 2018-04-01  expires: 2021-03-31  usage: C
     trust: ultimate      validity: ultimate
[ultimate] (1). Mario Rossi <mario.rossi@example.com>

gpg>
\end{lstlisting}

Dare il comando \texttt{addkey}:

\begin{lstlisting}
Please select what kind of key you want:
   (3) DSA (sign only)
   (4) RSA (sign only)
   (5) Elgamal (encrypt only)
   (6) RSA (encrypt only)
   (7) DSA (set your own capabilities)
   (8) RSA (set your own capabilities)
  (10) ECC (sign only)
  (11) ECC (set your own capabilities)
  (12) ECC (encrypt only)
  (13) Existing key
Your selection?
\end{lstlisting}

Scegliere \texttt{8}:

\begin{lstlisting}
Possible actions for a RSA key: Sign Encrypt Authenticate
Current allowed actions: Sign Encrypt

   (S) Toggle the sign capability
   (E) Toggle the encrypt capability
   (A) Toggle the authenticate capability
   (Q) Finished

Your selection?
\end{lstlisting}

Togliere la capacità di cifratura premendo \texttt{E}:

\begin{lstlisting}
Possible actions for a RSA key: Sign Encrypt Authenticate
Current allowed actions: Sign

   (S) Toggle the sign capability
   (E) Toggle the encrypt capability
   (A) Toggle the authenticate capability
   (Q) Finished

Your selection?
\end{lstlisting}

Scegliere \texttt{Q}:

\begin{lstlisting}
RSA keys may be between 1024 and 4096 bits long.
What keysize do you want? (2048)
\end{lstlisting}

Premere \texttt{Invio} per accettare \texttt{2048}:

\begin{lstlisting}
Requested keysize is 2048 bits
Please specify how long the key should be valid.
         0 = key does not expire
      <n>  = key expires in n days
      <n>w = key expires in n weeks
      <n>m = key expires in n months
      <n>y = key expires in n years
Key is valid for? (0)
\end{lstlisting}

Inserire \texttt{1y} per farla scadere dopo 1 anno:

\begin{lstlisting}
Key expires at lun 01 apr 2019 19:57:16 CEST
Is this correct? (y/N)
\end{lstlisting}

Accettare inserendo \texttt{Y}:

\begin{lstlisting}
Really create? (y/N)
\end{lstlisting}

Accettare inserendo \texttt{Y}.

Inserire quindi la passphrase e la sottochiave sarà aggiunta al portachiavi:

\begin{lstlisting}
We need to generate a lot of random bytes. It is a good idea to perform
some other action (type on the keyboard, move the mouse, utilize the
disks) during the prime generation; this gives the random number
generator a better chance to gain enough entropy.
\end{lstlisting}


Attendiamo che finisca, quindi apparirà:

\begin{lstlisting}
sec  rsa4096/4B6E6777
     created: 2018-04-01  expires: 2021-03-31  usage: C
     trust: ultimate      validity: ultimate
ssb  rsa2048/9114F367
     created: 2018-04-01  expires: 2019-04-01  usage: S
[ultimate] (1). Mario Rossi <mario.rossi@example.com>

gpg>
\end{lstlisting}


Aggiungiamo ora con la stessa procedura (partendo da \texttt{addkey}) la chiave
di cifratura e poi quella di autenticazione. Alla fine, dopo aver aggiunto la
sottochiave di autenticazione, dare il comando:

\begin{lstlisting}
save
\end{lstlisting}

per salvare e uscire.

Alla fine avremo questo portachiavi:

\begin{lstlisting}
gpg --list-secret-keys
\end{lstlisting}


\begin{lstlisting}
/home/mariorossi/.gnupg/pubring.kbx
-----------------------------
sec   rsa4096/4B6E6777 2018-04-01 [C] [expires: 2021-03-31]
uid         [ultimate] Mario Rossi <mario.rossi@example.com>
ssb   rsa2048/9114F367 2018-04-01 [S] [expires: 2019-04-01]
ssb   rsa2048/2F3AC08A 2018-04-01 [E] [expires: 2019-04-01]
ssb   rsa2048/AFBE0F10 2018-04-01 [A] [expires: 2019-04-01]
\end{lstlisting}

La situazione è, quindi, la seguente:

\begin{itemize}
    \item \emph{chiave principale}, adibita alla sola \emph{certificazione e
    gestione del portachiavi}: \newline \texttt{sec   rsa4096/4B6E6777 2018-04-01
    [C]};
    \item \emph{sottochiave di firma}: \newline \texttt{ssb   rsa2048/9114F367
    2018-04-01 [S]};
    \item \emph{sottochiave di cifratura}: \newline \texttt{ssb
    rsa2048/2F3AC08A 2018-04-01 [E] [expires: 2019-04-01]};
    \item \emph{sottochiave di autenticazione}: \newline \texttt{ssb
    rsa2048/AFBE0F10 2018-04-01 [A] [expires: 2019-04-01]}.
\end{itemize}

Possiamo aggiungere anche altri UserID o una foto con:

\begin{lstlisting}
gpg --expert --edit-key 0x4B6E6777
\end{lstlisting}

e poi con \texttt{adduid}.

\subsection{Il certificato di revoca}
\label{certificato-revoca}

Con GnuPG versione 2 la generazione del certificato di revoca è stata fatta
automaticamente quando le chiavi sono state generate e si trova nella
directory \texttt{/home/user/.gnupg/openpgp-revocs.d/}. Il file ha lo
stesso nome dell'impronta della chiave, nel nostro caso:

\begin{lstlisting}
491532759708014725CFE3A79F676B5A4B6E6777.rev
\end{lstlisting}

Questo file andrà tolto da questa directory e conservato in un luogo sicuro. Lo
vedremo nel paragrafo \vref{backup-certificato-revoca} dedicato al backup.

\subsection{Impostare le preferenze della chiave}

Modificare la chiave con:

\begin{lstlisting}
gpg --expert --edit-key 0x4B6E6777
\end{lstlisting}

Dal prompt di Gnupg dare:

\begin{lstlisting}
setpref S10 S9 S8 H10 H9 H8 Z2 Z3 Z1 Z0
\end{lstlisting}

Per il significato dei codici, vedi la Tabella \vref{table:codici}.

\begin{table}
   \centering
	\begin{tabularx}{6cm}{cl}
 		\toprule
		\textsc{Codice} & \textsc{Descrizione} \\
		\midrule
		\texttt{S10} & Cifrario Twofish \\
		\texttt{S9}  & Cifrario AES256 \\
		\texttt{S8}  & Cifrario AES192 \\
		\texttt{H10} & Hash SHA512 \\
		\texttt{H9}  & Hash SHA384 \\
		\texttt{H8}  & Hash SHA256 \\
		\texttt{Z2}  & Compressione ZLIB \\
		\texttt{Z3}  & Compressione BZIP2 \\
		\texttt{Z1}  & Compressione ZIP \\
		\texttt{Z0}  & Non compresso \\
		\bottomrule
	\end{tabularx}
	\caption{I codici GnuPG per cifrari, hash e compressione}
	\label{table:codici}
\end{table}

Alla domanda di GnuPG seguente:

\begin{lstlisting}
Set preference list to:
     Cipher: TWOFISH, AES256, AES192, 3DES
     Digest: SHA512, SHA384, SHA256, SHA1
     Compression: ZLIB, BZIP2, ZIP, Uncompressed
     Features: MDC, Keyserver no-modify
Really update the preferences? (y/N)
\end{lstlisting}

premere \texttt{Y} e quindi:

\begin{lstlisting}
save
\end{lstlisting}

\subsection{Impostare il file gpg.conf}

Aprire il file \texttt{gpg.conf} dentro la directory \texttt{/home/user/.gnupg},
eliminare tutto il contenuto e incollarvi questo testo, ricordandosi di
\emph{aggiornare le due \textsc{id} di chiave} nelle righe
\texttt{default-key} e \texttt{encrypt-to} con l'\textsc{id} della vostra
chiave:

\begin{lstlisting}
# Per lo skeleton di questo file vedi:
# /usr/share/gnupg2/gpg-conf.skel

#-----------------------------
# default key
#-----------------------------

# The default key to sign with. If this option is not used, the default key is
# the first key found in the secret keyring

default-key 0x9F676B5A4B6E6777
encrypt-to  0x9F676B5A4B6E6777

#-----------------------------
# behavior
#-----------------------------

# Disable inclusion of the version string in ASCII armored output
no-emit-version

# Disable comment string in clear text signatures and ASCII armored messages
no-comments

# Display long key IDs
keyid-format 0xlong

# List all keys (or the specified ones) along with their fingerprints
with-fingerprint

# List all keys (or the specified ones) along with their keygrips
with-keygrip

# Display the calculated validity of user IDs during key listings
list-options show-uid-validity
verify-options show-uid-validity

# Try to use the GnuPG-Agent. With this option, GnuPG first tries to connect to
# the agent before it asks for a passphrase.
use-agent

#-----------------------------
# keyserver
#-----------------------------

# When using --refresh-keys, if the key in question has a preferred keyserver
# URL, then disable use of that preferred keyserver to refresh the key from
keyserver-options no-honor-keyserver-url

# When searching for a key with --search-keys, include keys that are marked on
# the keyserver as revoked
keyserver-options include-revoked

#-----------------------------
# algorithm and ciphers
#-----------------------------

# List of personal digest preferences. When multiple digests are supported by
# all recipients, choose the strongest one
personal-cipher-preferences TWOFISH AES256 AES192 3DES

# List of personal digest preferences. When multiple ciphers are supported by
# all recipients, choose the strongest one
personal-digest-preferences SHA512 SHA384 SHA256 SHA224

# Message digest algorithm used when signing a key
cert-digest-algo SHA512

# List of personal compression preferences.
personal-compress-preferences ZLIB BZIP2 ZIP Uncompressed

# This preference list is used for new keys and becomes the default for
# "setpref" in the edit menu
default-preference-list SHA512 SHA384 SHA256 SHA224 TWOFISH AES256 AES192 3DES ZLIB BZIP2 ZIP Uncompressed
\end{lstlisting}

\subsection{Impostare il keyserver}

Modificare (o creare) il file \texttt{/home/user/.gnupg/dirmngr.conf}:

\begin{lstlisting}
# Per lo skeleton di questo file vedi:
# /usr/share/gnupg2/dirmngr-conf.skel

# This is the server that --recv-keys, --send-keys, and --search-keys will
# communicate with to receive keys from, send keys to, and search for keys on
keyserver hkps://hkps.pool.sks-keyservers.net
\end{lstlisting}

\subsection{Copiare la directory \texttt{.gnupg} nel proprio computer di lavoro}

A questo punto, se avete usato una distribuzione \textit{live} per la
generazione delle chiavi, copiate tutta la directory \texttt{.gnupg} su una
chiavetta \textsc{usb} e quindi trasferitela sul disco fisso del computer di
lavoro.

Dopo il trasferimento non dimenticate di distruggere la directory
\texttt{.gnupg} sulla chiavetta. Si possono usare diversi strumenti come ad
esempio \texttt{shred}.

\subsection{Esportare la chiave pubblica sul keyserver}

\begin{lstlisting}
gpg --send-keys 0x9F676B5A4B6E6777
\end{lstlisting}

\subsection{Segnarsi la data di scadenza delle proprie chiavi nel calendario}

Segnatevi nel vostro calendario quando scadranno le chiavi e fatevi mandare un
avviso almeno un mese prima. Una volta rinnovate le chiavi, inviatele al
keyserver. Ricordate che le chiavi scadute possono essere ancora rinnovate, ma
non è buona norma farle scadere.
