\section{Importare la chiave principale} \label{importare-chiave}

Quando periodicamente dovrete fare operazioni occasionali sul vostro portachiavi
o se dovete firmare (certificare) una chiave altrui, vi servirà la chiave
principale. Per fare queste operazioni dovrete importare la chiave principale
nel vostro portachiavi, fare l'operazione e quindi cancellarla nuovamente.

C'è un altro metodo per fare uso della chiave principale, come ad esempio dire
momentaneamente a GnuPG che la sua \emph{home} non è \texttt{/home/user/.gnupg}
ma un'altra directory. Io non ho mai usato questo sistema, per cui scrivo solo
quello che uso io.

Accertatevi di avere accesso dal terminale alla directory dove c'è il vostro
backup. Lo scopo è quello di copiare il file \texttt{.key} della chiave master
dal backup alla directory attuale di lavoro di GnuPG.

\begin{lstlisting}
cp directory-backup-gnupg/private-keys-v1.d/C6D1CD1D12CBBC8FA14D30004EAF381803A72597.key /home/user/.gnupg/private-keys-v1.d/C6D1CD1D12CBBC8FA14D30004EAF381803A72597.key
\end{lstlisting}

Effettuare quindi le operazioni che dovevamo compiere e, alla fine, eliminarla
nuovamente:

\begin{lstlisting}
cp rm /home/user/.gnupg/private-keys-v1.d/C6D1CD1D12CBBC8FA14D30004EAF381803A72597.key
\end{lstlisting}

Accertatevi ora che la chiave master non ci sia più con:

\begin{lstlisting}
gpg --list-secret-keys
\end{lstlisting}

che dovrebbe restituire qualcosa di simile a questo:

\begin{lstlisting}
sec#  rsa4096/0x9F676B5A4B6E6777 2018-04-01 [C] [expires: 2021-03-31]
      Key fingerprint = 4915 3275 9708 0147 25CF  E3A7 9F67 6B5A 4B6E 6777
      Keygrip = C6D1CD1D12CBBC8FA14D30004EAF381803A72597
uid                   [ultimate] Mario Rossi <mario.rossi@example.com>
ssb>  rsa2048/0xE7E0CAF69114F367 2018-04-01 [S] [expires: 2019-04-01]
      Keygrip = 135159EDEFB9FCB6062F9DE0E21FD90CB07DA041
ssb>  rsa2048/0x3C91B3682F3AC08A 2018-04-01 [E] [expires: 2019-04-01]
      Keygrip = C74D47329D5D224B0716879DCF3A4EC2D8951C53
ssb>  rsa2048/0x465ED456AFBE0F10 2018-04-01 [A] [expires: 2019-04-01]
      Keygrip = 30DEE8A060F3562CD6F1384E0F883D65477E596A
\end{lstlisting}

o a questo, se avete deciso di tenere le sottochiavi nel disco:

\begin{lstlisting}
sec#  rsa4096/0x9F676B5A4B6E6777 2018-04-01 [C] [expires: 2021-03-31]
      Key fingerprint = 4915 3275 9708 0147 25CF  E3A7 9F67 6B5A 4B6E 6777
      Keygrip = C6D1CD1D12CBBC8FA14D30004EAF381803A72597
uid                   [ultimate] Mario Rossi <mario.rossi@example.com>
ssb   rsa2048/0xE7E0CAF69114F367 2018-04-01 [S] [expires: 2019-04-01]
      Keygrip = 135159EDEFB9FCB6062F9DE0E21FD90CB07DA041
ssb   rsa2048/0x3C91B3682F3AC08A 2018-04-01 [E] [expires: 2019-04-01]
      Keygrip = C74D47329D5D224B0716879DCF3A4EC2D8951C53
ssb   rsa2048/0x465ED456AFBE0F10 2018-04-01 [A] [expires: 2019-04-01]
      Keygrip = 30DEE8A060F3562CD6F1384E0F883D65477E596A
\end{lstlisting}


C'è il simbolo \texttt{\#} dopo la parola \texttt{sec}, per cui la chiave master
non c'è più.
